\subsection{Educational Survey}
	If the player played one of these games, they are also required to take an educational survey before and after playing the game.
	\subsubsection{Darfur is Dying}
		The format of the quiz is 10 multiple choice questions, listed below. The same questions will be used for the pre and post survey, but the order will be randomized, and the multiple choice options will be reordered.
		\paragraph{Sources} \url{http://www.proprofs.com/quiz-school/story.php?title=what-do-you-know-about-darfur-genocide}, \url{http://www.funtrivia.com/playquiz/quiz2560151d4fc10.html}

		\paragraph{Darfur is in what region?}
			\begin{enumerate}
				\item Western region of Sudan
				\item Eastern region of Sudan
				\item Western region of Chad
				\item Eastern region of Chad
			\end{enumerate}

		\paragraph{Through the genocide in Darfur how many have been estimated of being displaced from their homes?}
			\begin{enumerate}
				\item 2.7 million
				\item 3.2 million
				\item 3.3 million
				\item 2.5 million
			\end{enumerate}

		\paragraph{How many civilians of Darfur have been murdered?}
			\begin{enumerate}
				\item 450,000
				\item 300,000
				\item 400,000
				\item 370,000
			\end{enumerate}

		\paragraph{What is said to be the main cause of the conflict?}
			\begin{enumerate}
				\item Religion
				\item Ethnic and tribal
				\item Power
				\item Economic differentiation 
			\end{enumerate}

		\paragraph{What year did the rebellion begin in Darfur?}
			\begin{enumerate}
				\item 2003
				\item 2002
				\item 2004
				\item 2005
			\end{enumerate}

		\paragraph{What has been the number one killer to civilians in Darfur}
			\begin{enumerate}
				\item Disease
				\item Starvation
				\item Homicide
				\item Suicide
			\end{enumerate}

		\paragraph{How is the region in Darfur divided?}
			\begin{enumerate}
				\item 4 federal states
				\item 3 federal states
				\item 2 federal states
				\item 1 federal state
			\end{enumerate}

		\paragraph{What group is referred to as “Devils on horseback?”}
			\begin{enumerate}
				\item Sunnis
				\item Shiite
				\item Tumoils
				\item Janjaweed
			\end{enumerate}

		\paragraph{What group are mostly Arab militias, supported by the government of Sudan, who conduct vicious attacks largely against non-Arab Darfurians.}
			\begin{enumerate}
				\item Sunnis
				\item Shiite
				\item Tumoils
				\item Janjaweed
			\end{enumerate}

		\paragraph{How many other Darifiran rebel groups are struggling to find unity and a common negotiating position to bring up against the government of Sudan, in spite of recent unity preceding the Libyan peace talks?}
			\begin{enumerate}
				\item 10
				\item 30
				\item 20
				\item 5
			\end{enumerate}

	\subsubsection{Oregon Trail}
		The format of the quiz is 10 multiple choice questions, listed below. The same questions will be used for the pre and post survey, but the order will be randomized, and the multiple choice options will be reordered.
		\paragraph{Sources} \url{http://library.thinkquest.org/J001587/}, \url{http://www.quizmoz.com/quizzes/Interesting-Facts-Quizzes/o/Oregon-Trail-Quiz.asp}, \url{http://www.quia.com/quiz/462983.html?AP_rand=227719871}

		\paragraph{How many miles long was the Oregon Trail?}
			\begin{enumerate}	
				\item 3000 miles
				\item 1000 miles
				\item 500 miles
				\item 2000 miles
				\item 20000 miles
			\end{enumerate}

		\paragraph{What is the name of a disease like malaria that the pioneers might catch?}
			\begin{enumerate}
				\item ague
				\item pneumonia
				\item epilepsy
				\item measles
			\end{enumerate}

		\paragraph{What year was the Oregon Trail first opened?}
			\begin{enumerate}
				\item 1847
				\item 1843
				\item 1899
				\item 1876
			\end{enumerate}

		\paragraph{What disease killed more people on the trail than any other?}
			\begin{enumerate}
				\item small pox
				\item plague
				\item cholera
				\item scarlet fever
			\end{enumerate}

		\paragraph{When was the first transcontinental railroad built that eventually ended the Oregon Trail?}
			\begin{enumerate}
				\item 1867
				\item 1870
				\item 1899
				\item 1869
			\end{enumerate}

		\paragraph{How many modern states did the travelers travel through when crossing the trail?}
			\begin{enumerate}
				\item 6
				\item 8
				\item 3
				\item 10
			\end{enumerate}

		\paragraph{How many people died on the Oregon Trail?}
			\begin{enumerate}
				\item 50,000-60,000
				\item 20,000-30,000
				\item 90,000-100,000
				\item 10,000-20,000
			\end{enumerate}

		\paragraph{Where did the Oregon Trail begin?}
			\begin{enumerate}
				\item Independence, Mississippi
				\item Independence, Missouri
				\item Independence, Michigan
				\item Independence, Montana
				\item Independence, Massachusetts
			\end{enumerate}

		\paragraph{Where did the Oregon Trail end?}
			\begin{enumerate}
				\item Vancouver, Washington
				\item The Dalles, Oregon
				\item Portland, Oregon
				\item Stevenson, Washington
				\item Oregon City, Oregon
			\end{enumerate}

		\paragraph{How many people came west on the Oregon Trail?}
			\begin{enumerate}
				\item at least 80,000
				\item at least 1,000,000
				\item at least 1,000
				\item at least 8,000
				\item at least 8,000,000
			\end{enumerate}

	\subsubsection{Light Bot}
		\paragraph{Sources} \url{http://www.cs.iastate.edu/~honavar/JavaNotes/Notes/chap16/chap16quiz.html}
		\paragraph{Functions}
			\begin{lstlisting}
				function foo() {
					bar();
					print('foo');
				}
				function bar() {
					print('bar');
				}
			\end{lstlisting}

for print('fizz'), write \"fizz\" (no quotes) in the text box. for print('fizz') followed later in the program by print('buzz'), write \"fizzbuzz\" in the text box.

questions: 4 pre, 4 post. random permutations of 4 foo or bar function calls strung together.

		\paragraph{Examples}

foo()
bar()
bar()
foo()

or:

bar()
bar()
bar()
bar()

		\paragraph{Loops} using 2 of the following questions for pre and the other 2 for post

4. Examine the following code:
	\begin{lstlisting}
		int count = 0;                                  
		while ( count <= 6 )  
		{
		  System.out.print( count + " " );
		  count = count + 2; 
		}
		System.out.println(  );
	\end{lstlisting}
What does this code print on the monitor?
\begin{enumerate}
	\item 1 2 3 4 5 6
	\item 0 2 4 6 8
	\item 0 2 4
	\item 0 2 4 6
\end{enumerate}

5. Examine the following code:
	\begin{lstlisting}
		int count = 7;                                  
		while ( count >= 4 )  
		{
		  System.out.print( count + " " );
		  count = count - 1; 
		}
		System.out.println(  );
	\end{lstlisting}
What does this code print on the monitor?
a. 1 2 3 4 5 6 7
b. 7 6 5 4
c. 6 5 4 3
d. 7 6 5 4 3

6. Examine the following code:
\begin{lstlisting}
	int count = -2 ;                                  
	while ( count < 3 )  
	{
	  System.out.print( count + " " );
	  count = count + 1; 
	}
	System.out.println(  );
	\end{lstlisting}
What does this code print on the monitor?
a. -2 -1 1 2 3 4
b. -2 -1 1 2 3
c. -3 -4 -5 -6 -7
d. -2 -1 0 1 2

7. Examine the following code:
	\begin{lstlisting}
	int count =  1;                                  
	while ( count < 5 )  
	{
	  System.out.print( count + " " );
	}
	System.out.println(  );
	\end{lstlisting}
What does this code print on the monitor?
a. 1 2 3 4
b. 1 2 3 4 5
c. 2 3 4
d. 1 1 1 1 1 1 1 1 1 1 1 . . . .



		\paragraph{Conditionals} procedural generation

\begin{lstlisting}
X: for each instance, randomly choose between True or False
%%: for each instance, randomly choose between 'and' or 'or'
#: for each instance, randomly choose between \'not\' or blank

format:
#(#X %% #X) %% #(#X %% #X)

for pre and post, generate 4 questions each

Does this conditional evaluate to True or False? (multiple choice)

\end{lstlisting}

	\subsubsection{Number Munchers}
	\paragraph{} 8 randomly generated questions for pre and post, 8th grade math

2 add, 2 sub, 2 div, 2 mult

