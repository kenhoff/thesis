\documentclass[12pt]{report}

\usepackage[margin=1in]{geometry}

\usepackage{url}
\usepackage[]{graphicx}
\usepackage{listings}

\author{Ken Hoff \\ University of Colorado at Boulder \\ \texttt{kendall.hoff@colorado.edu} 
% \\ \\ Advisor \\	\\ Alexander Repenning, Professor, University of Colorado 
% \\ \\ Committee \\	\\ Alexander Repenning, Professor, University of Colorado
% 					\\ Michael Eisenberg, Professor, University of Colorado
% 					\\ John Bennett, Professor, University of Colorado
% 					\\ Julien Lynge, Research Faculty, NOAA
}
\date{\today}
\title{Improving Educational Game Design Methods: A Rubric to Assess the Engagement and Educational Value of Educational Games
}

\begin{document}

\maketitle

\begin{abstract}

Existing educational games lack the combination of education and engagement. The objective of this thesis was to research, synthesize, and test a more effective educational design method. The procedure was to research current literature and existing effective educational games, synthesize an educational game design rubric based off of this research, then have apply this rubric to known educational games. 

Amazon's Mechanical Turk will be used to gather large amounts of data about the consistency of this rubric, after which the data will be collated and analyzed.

\end{abstract}

\tableofcontents
\listoffigures

\chapter{Introduction}
	Educational games lack the combination of education and engagement. In ineffective games, the content being taught is either uninteresting or unrelated to the gameplay. A good example is Math Blaster; players answer simple arithmetic questions to fill up their energy, which they then use to shoot space debris. These “games” are usually just homework problems, either included as an unrelated component inside the game, or interspersed between gameplay segments in a quiz format. It seems that game designers don't know how to produce effective and fun educational games. They are unaware or unfamiliar with the attributes their educational game should or shouldn't have in order to be as effective and fun as possible.

	\section{Thesis Statement/Solution}
		My goal in this research is to synthesize an educational game design rubric by creating a list of attributes that educational games should have a high degree of in order to be effective and engaging. The list is not a set of standards, but a set of guidelines; it is possible (though unlikely) for a game to contain all of the attributes and still be ineffective, or for a game to have none of the attributes and be very effective. It is simply meant to infer that game designs that contain these elements are more likely to be effective and engaging educational games.

		The question to be answered in this research is the following: Are there attributes that an educational game should have in order to be effective and engaging? In this work, I will develop a rubric of game attributes that educational games should have, and investigate whether some selected educational games incorporate these attributes. In addition, I will test the effectiveness of the selected games' ability to educate the players.

	\section{Approach} 
		I will begin by examining current educational game design rubricss, through research of published literature. I'll then synthesize my own educational game design rubric; a list of attributes found in effective and engaging educational games. Afterwards, I'll attempt to apply my rubric to existing educational and semi-educational games, and see if the most effective educational games correlate to my design rubric. In contrast, ineffective educational games won't correlate at all to my design rubric.

	\section{Organization}
		The rest of this thesis is organized as follows:

		Chapter 2 contains background research relevant to this field of research. It explores existing educational frameworks and research that influenced the selection of design elements for the rubric.

		Chapter 3 contains the design rubric that games are scored on. It contains 13 game design elements, each of which have been synthesized from the resources in the Research chapter. Each game design element may be found in existing games, to varying degrees described herein.

		Chapter 4 contains background information on how our example games were selected, as well as a listing for each game in our study. The game's overview and educational content is described, and screenshots, sources, and notes on implementation are included for each.

		Chapter 5 contains all information related to the procedure of our study. It includes how Amazon's Mechanical Turk works, as well as design decisions for the content and process of administering the survey. It also includes the survey that was administered, as well as all permutations and additions.

		Chapter 6 contains the aggregated results of the administered survey, as well as non-statistical observations on the data by myself. It includes results from the game opinion, quiz, and rubric sections of the survey.

		Chapter 7 contains the statistical analysis of the prominent parts of the survey, the quiz and rubric. We use the t-distribution to determine the statistical significance of the quiz score differences, and use an inter-rater reliability metric to determine the cohesiveness of our design rubric.

		Chapter 8 contains conclusions that we've explored in the Results and Analysis sections, as well as a retrospective and future options for additional research.

\chapter{Research}
	\section{Edutainment vs Educational Games}

		It's important for us to differentiate between edutainment and educational games. For this paper, we're going to define edutainment as the simple gamification of a task. The purpose of edutainment is to increase the player's skill at something by getting them to play a game; the player mimics or actually does the skills within the game, and the game rewards them and keeps them engaged in order to continue increasing their skill. These kinds of games, while effective, are ineffective at teaching complex concepts and systems. For that purpose, we use educational games; they have the more rewarding task of teaching complicated content, but it is far more difficult to do so, as they can't utilize the “skill and drill” method of education. We're going to focus primarily on educational games in this paper.


	\section{Bloom's Taxonomy}

		Bloom's Taxonomy is one of the most widely cited classifications of learning. It provides milestones for three different domains of learning: Psychomotor, Affective, and Cognitive.

		The first domain, Psychomotor, deals with the ability to perform intricate physical tasks, like hammering a nail or operating a complex machine or instrument. Games are currently very good at teaching concepts within the Psychomotor domain. The best examples are ones where the game provides a near-simulation to the task being performed in real life. Music games like Guitar Hero and Rock Band improve coordination and motor control while playing an instrument. Simulation driving games like Forza and Gran Turismo provide players with continuous feedback to help them learn the correct response to stimulus given the car they are driving, to help them improve their driving skills. Other examples include any game that improves motor control; Kinect's Dance Central improves player's bodily movements, and even first person shooter games like Halo or Call Of Duty improve player's hand-eye coordination.

		The Affective domain is one that is far harder for games to address. It deals with learning empathy, emotion, and includes things that are commonly learned through interaction with people and not necessarily through formal education. Games commonly expose players to this in two ways. The first is through game narrative, where players are exposed to characters, situations, and choices where they observe and learn about the human condition. Games with extreme character development (Mass Effect, The Walking Dead) typically explore human choices in great detail. There are also extremely minimalistic games that communicate powerful empathic ideas without a large amount of gameplay or graphics, or conversation (Thomas was alone, Journey), as well as completely narrative-based, choose-your-own-adventure Twine games that address extremely empathic concepts (Howling dogs). The second is through putting players in situations with other players where they will need empathy and skills from the Affective domain in order to solve their problems. All multiplayer games facilitate this to some degree, but puzzle games (Portal, LittleBigPlanet), and MMO's (WoW, EVE Online) generally encourage players to communicate, negotiate, and work together to achieve a common goal.

		When we think of educational games, we commonly think of the Cognitive domain. The Cognitive domain deals with knowledge of facts, associations, and mechanics, and typically consists of the material that educators attempt to teach in schools. This is also the domain that games attempt to address the most; being able to communicate these concepts effectively would reduce the amount of material educators would need to teach, allowing them to focus on more complex concepts. Because games in the Psychomotor domain are well traversed, and games in the Affective domain are far more nebulous as to measuring their educational value, we will be focusing exclusively on the Cognitive domain.

			

	\section{Tangential Learning}

		In their paper on “Serious Games and Learning,” Breuer and Bente (2010) outline a method called “Tangential Learning.” Originally presented as part of the “Extra Credits” video series online (Floyd and Portnow, 2008), the presenter gives us a method for which we can encourage significant learning using games. The theory is that, given the proper incentive, players will engage in a form of self-education separate from the game environment.
		
		They base this on the principle that we're able to recall and understand material that we're more interested or passionate about, rather than material that is uninteresting or boring to us. The same principle applies to video games; if we are interested in the context of an educational game, we're far more likely to remember that instead of the content being taught as part of it. The example Extra Credits uses is geography; it's easy for some players to draw the entire map of Azeroth from memory, but they're unable to properly identify US states.
		
		The most prominent field that this could be applied to is history; as most games are set in the past, such as Ancient Rome or World War II, it would be easy for players to learn more about history on their own simply because they were playing the game in the appropriate context. There are other contexts that players can be placed in, in order to educate about other topics. Players in music games can self-educate on music theory and different musical genres, and players in space simulations might self-educate on spaceflight and astronomy. 
		
		However, there are numerous other ways players can be educated apart from being placed in the context of the educational goal. An easy-to-implement way would be to include facts as part of the loading screens in the game; instead of having game tips or strategies, designers could include interesting bits of external information to help players retain information. Also, designers can include subtle references to real-life objects within their games. Extra Credits gives the example of Sephiroth from Final Fantasy, or the Excalibur and the Masamune. By naming characters or objects after real-life characters or objects, designers can encourage players to go out and seek the origin of the character or object; it also helps to have once object in a group be something that's easily recognizable, so that players immediately know that other objects in the group are referential to the real world.
		
		Another excellent way to integrate tangential learning into games is by including an in-game encyclopedia, where players don't even have to leave the game to explore more about the topic they're interested in. Usually, these encyclopedias include information related to playing the game; the stats of an object, or the rules of the game. However, some include information that isn't related to playing the game, like the history or origin of an object or character. A notable example of this is the game Civilization, which includes the Civilopedia, a reference to every unit, building, and wonder in the game, as well as other information related to it. An easy way for games to integrate this would be just including Wikipedia links to every object in the game that has real world roots.
	
	\section{Flow}

		Dondlinger (2007) explains why it's extremely important to give players the perception of free will. In educational games, we don't want to give the players no direction as to where to go within the game; we risk players missing the educational goals, and worse, becoming frustrated with the lack of progress. Conversely, we don't want to overly guide the players; too much hand-holding results in the game stifling creativity, as well as not allowing players to really “learn by doing.” It's important for us to give them the illusion of having an open world, where almost any actions are possible, in order to allow players to experiment and learn by doing.

		In parallel to the illusion of free will is the concept of ‘flow.' In games, we want to the player to experience just the right difficulty; games that are too hard will be frustrating, and games that are too easy will be boring. The same is true of education and educational games; material that is far beyond the comprehension of the learner won't be retained, but material that the learner has already covered sufficiently will be boring. We need to maintain a level of difficulty to our game that Csikszentmihalyi (1997) calls ‘flow.' When players are in a state of ‘flow,' the game provides to them a clear set of inputs and outputs; material that they are somewhat familiar with, but also obscure enough to encourage to players to try several options in order to solve the problem.

		Following the concept of ‘flow,' we can extrapolate the concept of ‘adaptive difficulty.' It's logical to assume that games that continually adapt their difficulty level to the players will be more effective in retaining the player's interest and educating the player than games that have a fixed difficulty curve. However, not every game can implement adaptive difficulty effectively; games need the players to continuously attempt problems and give feedback on how well they're doing to ascertain how well the difficulty matches up to the player.

	\section{Play}

		Similar to the illusion that players have a seemingly limitless game world to interact with is the concept of ‘play.' Paras (2005) defines a world where play can happen simply as a world with a series of constraining rules. We don't want the rules to be too constraining, because then we inhibit creativity and discourage alternative solutions to problems, but at the same time, we want the rules to be constraining enough to guide the player towards the solution and mimic the rules of the real world.
		
		The way the game guides players around these rules is extremely important. If the game gives the player copious amounts of tutorials and guidance as to what they can or can't do in the game, the player won't learn them as well. This isn't to say that the player should be encouraged to break the game rules, but instead to encourage the player to explore the world's rules; they won't understand what will and won't work unless they encounter it on their own and are enabled to learn from it.
	
		The game also shouldn't penalize the player for exploring the boundaries of the game world, as it would discourage them from trying new things and discovering additional boundaries. However, we still need to encourage players to solve problems within the game world without breaking any boundaries, just to ensure that they understand the world that they're playing in; that's why rewarding the players for not breaking any rules works much better than penalizing the player for breaking rules.

\chapter{Rubric}
	What follows is a list of properties that make up the educational game design, created by researching and synthesizing properties of existing educational games. Games are not required to have all or any of these properties, but the theory is that games with more of these properties will be considered more effective and engaging educational games.

\input Rubric

\chapter{Games}
	\section{Game Selection}
		In order to test our rubric, we have to have testers play some educational games. We looked for games marketed as educational games, as well as games that could have been considered educational by including tangential learning. We selected the games from existing knowledge, as well as from references in existing literature.

		\subsection{Web games}
			Due to the EULA of Mechanical Turk, we could not ask Workers to install any software on their machines as part of our task. This meant that we couldn't select some popular educational games, like Typing of the Dead, because users would have to install the game on their machine.

			An option that was considered was having Workers find a YouTube or other video of the game being played, and try to evaluate the game's properties based on the gameplay footage. However, we decided that Workers wouldn't be able to evaluate some of the game's properties (e.g. penalties for failure, or contextual tutorials) based on a gameplay video. 

			Thus, our primary selection criteria for educational games was that they had to be web-based. The players could not be required to install any software onto their machine in order to play the game. This left us with many interesting and high-quality options, like Light-Bot and Darfur is Dying, but it eliminated some other high-quality options like Logical Journey of the Zoombinis.

			It's important to note that we were allowed (or at least, didn't run into any problems) to ask users to install web browser plugins in order to play games. Some of these plugins were commonplace (like the Java runtime), but others were obscure (like the Apple II emulator). This allowed us to find some popular educational games usually only available as a desktop applications, like The Oregon Trail or The Incredible Machine, even if they were very old versions of the games.

		\subsection{Math Blaster}
			One of the most prominent games that we were unable to include in our survey was the Math Blaster series. Math Blaster's The Search For Spot, released in 1994, was the earliest version of the game, and the franchise continues strong to this day. More editions and expansions were released, all as installable desktop applications.

			Recently, the Math Blaster franchise has transitioned into a web-based format, where players can play various Math Blaster games online. This would have been perfect for our study, but the games exist behind a `login wall,' where players must create accounts and verify their email addresses before they are allowed to play the games. We assume this violates the EULA of Mechanical Turk, but the login wall presents such an accessibility obstacle for Workers that we decided to omit Math Blaster altogether. 

		\subsection{Where in the World is Carmen Sandiego?}
			Another extremely popular and highly-acclaimed educational game series, Where in the World is Carmen Sandiego, started releasing desktop games in 1985, but has since ceased producing sequels. Online, we found numerous emulators for the original game, but the performance and framerate for each of them was very poor. We decided that the average Worker would find it unplayable, and decided to omit it from the study.

		\subsection{Old Games}
			It's important to note that when we think of good, educational games, we commonly think of classic games; games that were created early in the lifecycle of educational games, and were highly acclaimed for their educational content. Recently, there haven't been many games that have been popularized for their educational content.

			Why is that? There's a couple ways that we could explain the lack of popular educational games. It's possible that there exist as many or a greater number of excellent educational games presently, but due to saturation of the gaming industry (both as entertainment and educational games), educational games don't get as much visibility as they do before. For example, the most popular recent game included in our survey, Darfur is Dying, had over 800,000 plays in 6 months and was covered in the media, but received no `educational game' awards.

			Another option is that the nature of education in gaming is changing. Instead of players seeking deliberately edcuational games, where education is the primary goal of the game, players may be more interested in games where education is the secondary goal. Games with education as the secondary goal most likely use tangential learning in order to teach players without deliberately lecturing or quizzing them. Light-bot, one of the games in our survey, is a great example of this.
	\section{Game Listings}

\input GameSummaries

\input survey

\input results

\input analysis
	


\chapter{Conclusion}
	% At present, educational games do not actually increase the content knowledge of their players. Across 80 surveys, the average score of a pre- and post-quiz increased 3\%; not a statistically significant amount. Further measures can be taken to increase the quality of the survey and quiz, the effectiveness of using Mechanical Turk as a platform for research, and focusing the research on one specific game with more responses.
	For our four games with quizzes (Darfur is Dying, The Oregon Trail, Light Bot, and Number Munchers), only The Oregon Trail and Number Munchers had statistically significant results. The Oregon Trail's was positive, with a mean improvement of 5\%. Number Munchers' was negative, with a mean difference of -1\%.

	For our design rubric, the highest rated four design elements (Adaptive Difficulty, Contextual Tutorials, Unorthodox Problem-solving, and Game Resource Penalty) fell in the category of ``Slight Agreement.'' For Adaptive Difficulty, Contextual Tutorials, and Unorthodox Problem-solving, the scores on each of the games were universally negative or mixed.

	For the rubric item ``Game Resource Penalty,'' three games (Notpron, Botlogic and Light Bot) had relatively positive scores, and three games (The Oregon Trail, Number Munchers, and Pandemic 2) had relatively negative scores.

	It's implied with small statistical confidence that, because The Oregon Trail had a statistically significant improvement in quiz scores, that Game Resource Penalty is something that workers agree on, that Game Resource Penalty varies from game to game, and that The Oregon Trail has a very poor Game Resource Penalty rating, we can say that an educational game should have the maximum amount of Game Resource penalty possible in order to be the most effective.

	Punishing players for failure seems very contradictory to existing research. This conclusion comes with numerous caveats; the small difference in quiz scores, the extremely small inter-rater agreement values, and other unknown game design elements contribute to this being a result of no confidence. There are many unknown variables; the format of the game, quiz, survey, and administration process can all contribute to inconclusive work.

	It is my opinion that we can derive no significant results from the data without further analysis. However, there are numerous learnings considering the methods we used to obtain these results, and how we can optimize them to create better results.

	\section{What worked well}
	\section{What didn't work well}
		There are a number of reasons why the results may have fallen so flat.

		The quiz itself may have been faulty; perhaps the questions were too hard, too easy, or covered content that was not taught as part of the game. To combat this, the quizzes were designed to cover material that players were supposed to learn in the game, but it's possible that the multiple-choice format wasn't the best way to do that.

		It's also possible that the games weren't actually effective in teaching the players anything at all. Because of how the games were designed, players didn't retain any relevant information about what the game was trying to educate them on.

		Another option is that the platform on which the surveys were administered has faults. There are spammers on Mechanical Turk, and while it is possible to review and reject responses deemed unusable, it's impractical to filter out every single one. However, if a spammer is selecting random responses, it's highly unlikely that they'll end up with a higher or lower score than when they started. 

		It's also possible to filter out poorly rated workers with an ``acceptance rate" criteria, where Workers that accept the task are required to have at least a 90\% acceptance rate across all of their HITs, but that increases response times and payout amounts.

		Finally, it's possible that some workers lost interest in the HIT once they had finished the first sections, resulting in shoddy work on the second. This would mean that some workers would guess at the questions in the post-quiz, resulting in worse performance across the board.
	\section{What I learned}
	\section{What to do next}
		For a future quiz/survey on the Mechanical Turk platform, I'd recommend both an ``acceptance rate" criteria, as well as some form of automated validation that the worker is actually completing the survey. This may include a ``dummy question," where the reviewer only has to click the right button, or some other kind of verification where they submit a screenshot of their highest score within the game. This will increase response times and require higher payout amounts, but if it's focused only on one game it won't be much.

		Another route worth exploring is paying out ``bonuses" to workers who show exemplary work in the quiz and survey. However, you'd have to make sure to incentivize honesty and not greed; my hunch is that asking workers to not use the internet to find solutions may not work when there's extra payouts involved.




\begin{thebibliography}{1}

\bibitem{} Breuer, Johannes, and Gary Bente. ``Why so Serious? On the Relation of Serious Games and Learning.'' Eludamos. Journal for Computer Game Culture 4.1 (2010): 7-24. Eludamos. Journal for Computer Game Culture. Web. 02 July 2013.
\bibitem{} Buhrmester, Michael. ``Amazon Mechanical Turk Guide for Social Scientists.'' Web. \textless \url{http://homepage.psy.utexas.edu/homepage/faculty/gosling/reprints/MTurkhowto.pdf}\textgreater.
% \bibitem{} Coe, Robert. ``It's the Effect Size, Stupid: What Effect Size Is and Why It Is Important.'' It's the Effect Size, Stupid: What Effect Size Is and Why It Is Important. British Education Index, 12 Sept. 2002. Web. 21 Nov. 2013.
\bibitem{} Csikszentmihalyi, Mihaly. ``Finding Flow.'' Psychology Today. 14 June 2012. Web. 02 July 2013.
\bibitem{} Dondlinger, Mary Jo. ``Educational Video Game Design: A Review of the Literature.'' Journal of Applied Educational Technology 4.1 (2007). Print.
\bibitem{} Floyd, Daniel, and James Portnow. ``Tangential Learning.'' Season 2, Ep. 9 - Tangential Learning. Web. 02 July 2013.
\bibitem{} ``Genocide in Darfur.'' Game. Fun Trivia. Web. 21 Nov. 2013. \textless \url{http://www.funtrivia.com/playquiz/quiz2560151d4fc10.html}\textgreater.
\bibitem{} Hallgren, Kevin A. ``Computing Inter-Rater Reliability for Observational Data: An Overview and Tutorial.'' NCBI. U.S. National Library of Medicine, 23 July 2012. Web. 21 Nov. 2013.
\bibitem{} Hyman, Paul. ``Communications of the ACM.'' Software Aims to Ensure Fairness in Crowdsourcing Projects. Communications of the ACM, Aug. 2013. Web. 21 Nov. 2013.
\bibitem{} Kjell, Bradley. ``Counting Loop Quiz.'' Counting Loop Quiz. Web. 21 Nov. 2013. \textless \url{http://www.cs.iastate.edu/~honavar/JavaNotes/Notes/chap16/chap16quiz.html}\textgreater.
\bibitem{} ``The Measurement of Inter-rater Agreement.'' Web. \textless \url{http://hpm.fk.ugm.ac.id/hpmlama/images/Biostatistik/Tutorial\_4\_AS/2-chapter18.pdf}\textgreater.
\bibitem{} Paras, Brad, and Jim Bizzocchi. ``Game, Motivation, and Effective Learning: An Integrated Model for Educational Game Design.'' Proc. of DiGRA 2005 Conference: Changing Views – Worlds in Play. 2 June 2005. Web. 2 July 2013.
\bibitem{} ``Pioneer Life.'' ThinkQuest. Oracle Foundation. Web. 21 Nov. 2013. \textless \url{http://library.thinkquest.org/J001587/}\textgreater.
\bibitem{} ``Quiz: The Oregon Trail.'' Quia. Quia. Web. 21 Nov. 2013. \textless \url{http://www.quia.com/quiz/462983.html?AP\_rand=227719871}\textgreater.
\bibitem{} ``QuizMoz - Oregon Trail Quiz.'' QuizMoz - Oregon Trail Quiz. Web. 21 Nov. 2013. \textless \url{http://www.quizmoz.com/quizzes/Interesting-Facts-Quizzes/o/Oregon-Trail-Quiz.asp}\textgreater.
\bibitem{} ``What Do You Know about the Darfur Genocide?'' What Do You Know about the Darfur Genocide? ProProfs. Web. 21 Nov. 2013. \textless \url{http://www.proprofs.com/quiz-school/story.php?title=what-do-you-know-about-darfur-genocide}\textgreater.

\end{thebibliography}


\end{document}